\appendix
\chapter{Definition of hit-level input variables}
\label{appendix:hit-training-var}
Throughout the GNN4ITk algorithm, machine learning models are trained using input features associated with 
the space points contained in a collision event. 
These features can be broadly grouped into those of the reconstructed space points, and those of the underlying clusters.
This appendix explains the meaning of these variables, listed in table \ref{tab:input-metric-learning} in the text.

% \begin{enumerate}[I]
%     \item Space point features \\ 
%     These features include the reconstructed position of the intersection between the true particle path and the detector layer. 
% The position is simply represented by its cylindrical coordinates $ (r,\phi,z)$.

%     \item Cluster features \\
%     \begin{enumerate}[i]
%         \item Charge count: $$C = \sum_i c_i,$$ where $c_i$ is the charge deposited in each cell in the cluster.
%         \item Cell count: Total number of cells contained in the cluster.
%         \item Direction of the cluster in the local coordinate: (localDirection + loc-phi + loc-eta)
%         \item LengthDirxyx
%         \item Direction of the cluster in the global coordinate: (glob_eta, glob_phi)
%     \end{enumerate}
% \end{enumerate}


\section{Cluster features}





